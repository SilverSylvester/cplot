\documentclass[a5paper,10pt]{article}
\usepackage[english]{babel}

% maths packages
\usepackage{mathtools,commath,amsthm,amssymb,thmtools}

% code
\usepackage{minted}

% Document-wide formatting
\usepackage{geometry}
\usepackage{enumitem}
\linespread{1.05}
\setlist{noitemsep}

% links
\usepackage{hyperref}
\usepackage{xcolor}
\hypersetup{
  colorlinks,
  linkcolor={red!50!black},
  citecolor={blue!50!black},
  urlcolor={blue!80!black}
}

% Fonts
\usepackage[utf8]{inputenc}
\usepackage[T1]{fontenc}
\usepackage{microtype}  % better spacing
\usepackage{libertine}
\usepackage[libertine]{newtxmath}

\title{Diary}
\author{Conor Reynolds}
\date{}

\begin{document}
    \maketitle

    \section*{2017-09-21}

    Pretty much decided to use the `pipes' library for data streaming. As for graphics packages, so far we have:

    % add links to these
    \begin{itemize}
        \item gloss
        \item dynamic-plot
        \item dynamic-graph
        \item Chart
        \item diagrams
        \item typed-spreadsheet
        \item plain OpenGL bindings
    \end{itemize}

    Going ahead with using gloss, mainly for ease of use (esp. since I have no prior background in any kind of graphics programming).

    \section*{2017-09-22}

    Decided to use Haskell's STM library to handle shared mutable state between gloss and the rest of the program. A gloss process is kicked-off with access to a TVar, which is modified by a pipe that continuously reads data from stdin (one line at a time). All it does, currently, is plot a stream of $ x,y $-coordinates onto a blank, white canvas.

    \section*{2017-09-24 --- 2017-09-25}

    On a new branch, \verb|anim->play| (which was, in hindsight, not a very good name for a branch), began the migration from the animateIO function to the playIO function to allow for custom event handling, which is no doubt needed. This lead to creating a barebones MVC-style backend, which should allow for much more complex features to be added in the future.

    \section*{2017-09-26}

    Merged \verb|anim->play| with \verb|master|. Work has begun on a new branch, \verb|axes|. This will deal primarily with drawing the axes, although it may end up involving writing a more robust view framework. Inside the viewport, you can have any number of `frames', which are distinct viewing areas, each having their own dataset, zoom level, styles, etc.

    \section*{2017-10-21}

    After lots of quiet experimentation, cplot now uses
    \begin{itemize}
        \item gtk3 and frpnow as GUI backend
        \item Chart library for defining charts
        \item Cairo for rendering charts to gtk3 widgets
    \end{itemize}
    Multiple other options were explored:
    \begin{itemize}
        \item External JS libs for charting (chart, plotly, epoch), communicating with a Haskell process via websockets
        \item HaskellR
        \item Some other arbitrary external process communicating via zeromq / nanomsg or some other messaging protocol
    \end{itemize}
    All had multiple problems that were more or less irreconcilable given the aim of this project, but importantly they were all a bit too slow for fast real-time data visualisation. The current approach (modulo a better system for managing large amounts of data) is vastly more efficient than any other approach attempted so far.
\end{document}
